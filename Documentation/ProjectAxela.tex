\documentclass[a4paper,notitlepage]{article}
\usepackage{nameref}
\usepackage{hyperref}
\hypersetup{
    colorlinks,
    %citecolor=black,
    %filecolor=black,
    linkcolor=black,
    %urlcolor=black
}
\usepackage{xcolor,colortbl}
\usepackage{graphicx}
\usepackage{color}

\usepackage{fancyhdr}
\setlength{\headheight}{1cm}
\setlength{\headwidth}{\textwidth}
\fancyhead[L]{}% empty left
\fancyhead[R]{ % right
   \includegraphics[height=1cm]{images/logo_UA_hor_kl_PMS_sec.pdf}
}
\pagestyle{fancy}




\begin{document}
\begin{titlepage}
    \centering
    \vfill
    {\bfseries\Large
        II-Ambient Intelligence: Low Power Embedded Communication\\2016-2017\\
        
    } 
    
    \hrulefill
    \vskip4cm
    
    {\bfseries\large
            Project Axela: The Documentation\\
            \vskip2cm         
    }   
    {\bfseries   
            
            Supervisors: Maarten Weyn\\
           	14/06/2017   
            
            
            
            
    } 
    
    \vfill
    \includegraphics[width=4cm]{images/logo_UA_hor_kl_PMS.pdf}     
\end{titlepage}

\tableofcontents
\vfill
\newpage


\section*{Project Overview}
\addcontentsline{toc}{section}{Project Overview}

The goal of this course is creating a Low Power Communication System using a Raspberry Pi 3 and some mobile nodes. On the Raspberry Pi, we're running an instance of Debian Jessie. The Raspberry will be our gateway for different mobile nodes. It will be a stationary device that has multiple sensors connected and it will receive data from the different mobile nodes. Different packages will be installed on the Raspberry to complete our project successfully, including OpenHAB.\\

OpenHAB is a software for integrating different home automation systems and technologies into one single solution that allows over-arching automation rules and that offers uniform user interfaces. It will show all the data from the different mobile nodes and sensors on the Raspberry.\\

The following design gives a detailed view on our project. It's show the used sensors and mobile nodes that have been designed solely for this project.

\textcolor{red}{ADD PICTURE}

\section*{Project Details}
\addcontentsline{toc}{section}{Project Details}

All the details on the project will be explained here, it's a small overview so everyone can follow what other people have done and how they implemented it.

\subsection*{Raspberry Pi 3}
\addcontentsline{toc}{subsection}{Raspberry Pi 3}

	\subparagraph*{Alexa}
	\addcontentsline{toc}{subsubsection}{Alexa}
	
	\subparagraph*{SI7021 Humidity Sensor}
	\addcontentsline{toc}{subsubsection}{SI7021 Humidity Sensor}
	
	\subparagraph*{MAG3110 Magnetometer}
	\addcontentsline{toc}{subsubsection}{MAG3110 Magnetometer}
	
	\subparagraph*{MPL3115A2 Barometer}
	\addcontentsline{toc}{subsubsection}{MPL3115A2 Barometer}
	
	\subparagraph*{CCS811 CO2 sensor}
	\addcontentsline{toc}{subsubsection}{CCS811 CO2 sensor}

\subsection*{STM-32 Board Sensors}
\addcontentsline{toc}{subsection}{STM-32 Board Sensors}

\subsection*{Dash 7 Implementation + fingerprinting}
\addcontentsline{toc}{subsection}{Dash 7 Implementation + fingerprinting}

\subsection*{OpenHAB}
\addcontentsline{toc}{subsection}{OpenHAB}

\end{document}