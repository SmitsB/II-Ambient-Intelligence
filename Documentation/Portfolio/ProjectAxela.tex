\documentclass[a4paper,notitlepage]{article}
\usepackage{nameref}
\usepackage{hyperref}
\hypersetup{
    colorlinks,
    %citecolor=black,
    %filecolor=black,
    linkcolor=black,
    %urlcolor=black
}
\usepackage{xcolor,colortbl}
\usepackage{graphicx}

\usepackage{fancyhdr}
\setlength{\headheight}{1cm}
\setlength{\headwidth}{\textwidth}
\fancyhead[L]{}% empty left
\fancyhead[R]{ % right
   \includegraphics[height=1cm]{images/logo_UA_hor_kl_PMS_sec.pdf}
}
\pagestyle{fancy}




\begin{document}
\begin{titlepage}
    \centering
    \vfill
    {\bfseries\Large
        II-Ambient Intelligence: Low Power Embedded Communication\\2016-2017\\
        
    } 
    
    \hrulefill
    \vskip4cm
    
    {\bfseries\large
            Project Axela: The Documentation\\
            \vskip2cm
            
            Bernd Smits\\
            Sebastiaan Aussems\\
            Kwinten Schram\\
            Frederik Smolders\\
            \vskip2cm           
    }   
    {\bfseries   
            
            Supervisors: Maarten Weyn\\
           	01/01/2017   
            
            
            
            
    } 
    
    \vfill
    \includegraphics[width=4cm]{images/logo_UA_hor_kl_PMS.pdf}     
\end{titlepage}


\section*{Abstract}
This proposal will be about my bachelor's project, the self-driving F1 car. I will need to continue the work done by previous students to eventually reach an end-goal, which is a autonomously driving car. In this proposal I state the steps I will take to complete my research. 

The main goal of my project is to create a car that knows where it is located and that it knows how to follow a certain route without colliding with any objects. To complete this task I will use data from different sensors like the LIDAR sensor (the same sensor used by Google in their self-driving car project).

\tableofcontents
\vfill
\newpage


\section*{Bachelor's project description}
\addcontentsline{toc}{section}{Bachelor's project description}

\subsubsection*{Problem Description}
\addcontentsline{toc}{subsection}{Problem Description}

My bachelor's project will be about the F1/10 car\cite{F1}. The F1/10 competition focuses on creating a challenging design experience for students. Students from all around the world will need to build software that can let a small self-driving Formula 1 car (scale 1/10th) compete against other cars.\\

For my bachelor project, I need to write software that enables the F1 car to autonomously drive as fast as possible around a track while avoiding collisions. I still have a long road ahead to achieve this goal but we get a little closer every time students work on it. \\

This means that the project has been started by other students before me and I will continue their work. I will start where they left of and continue their journey to the end-goal.

\subsubsection*{Research question / Thesis statement}
\addcontentsline{toc}{subsection}{Research question / Thesis statement}

Building an autonomously driving car is a very widespread project. It's better to divide this in multiple steps. I already know, from students who've worked on this project, that the car has been built with all the needed sensors. Furthermore, the car can already be controlled remotely by a computer using the arrow keys.\\

The next step to our end-goal is to let the car know it's own location and let it complete a given track without any obstacles (this means it can complete a previously known/constructed track). A step further is to let the car finish an unknown closed track with still no obstacles. If these steps are completed the detection of objects can be implemented.  

\subsubsection*{Approach / Methods}
\addcontentsline{toc}{subsection}{Approach / Methods}

As mentioned in the thesis statement I will be conducting research to create a car that drives autonomously. The basics of the project have already been finished. A colleague of mine has built the car that I will be using. He even finished the first step in our project to allow the car to be controlled remotely. Because he is still working on the project while writing this paper I can't link any of his work.\\

When my colleague finishes his work, I'll pick up where he left off. My first step is of course understanding the car itself, how is the data from the sensors captured, how can I control the car and how do the different modules on the car communicate with each other. \\
\newpage
Once I have the knowledge to continue I can start working on my real job, let the car know it's own location. This is a very important step later on to let the car drive itself. The organisation F1/10 provides me with a lecture on using the sensors to find my location\cite{F1local}. Further steps like driving the car on a straight line or following a track have also been explained in the lectures, these steps will be finished during my project.

\section*{Work plan including time table}
\addcontentsline{toc}{section}{Work plan including time table}

The following table gives a detailed view on how I want to proceed with my project. The first few weeks of the semester is meant for gathering information to speed up some stages along the way. Later, when the project indeed starts I start with creating a basic self-driving car.

\begin{table}[h]
\centering
\begin{tabular}{ c | p{3cm} | p{6cm} }
	
	{\bf{Week (2017)}} & {\bf{Deadline}} & {\bf{Deadline info}} \\
	\hline
	&&\\
	13-03 & Info Received & I get my official project subject and tasks \\
	20-03 & Info Processed & I understand the project \\
	27-03 & Stage 1 completed & I know how the car works with all the sensors and how it is remotely controlled \\
	03-04 & Stage 2 Start & The car needs to know it's location\\
	10-04 & Stage 2 continued & Follow the lectures to continue my research\\
	24-04 & Stage 2 completed & Finishing touches on stage 2 and starting stage 3\\
	01-05 & Stage 3 started & creating PID controller for car\\
	22-05 & Stage 3 Finished & created PID controller, basic self-driving car is finished\\ 
	07-06 & Present Project & Project needs to be finished and presented\\
\end{tabular}
\caption{Time Table}
\end{table}

\section*{Implications of research}
\addcontentsline{toc}{section}{Implications of research}

Overall this bachelor project won't be anything new to someone who has studied software engineering. This is a project around the basics of self-driving cars for students who are interested and want to learn something new in practice. This project is the perfect balance between electronics (using sensors) and writing software (the 'AI'). For me, as a student, this is perfect. I get to learn new things and not just the theory behind everything, I also get the practical background which is always preferable.

\section*{Contacts}
\addcontentsline{toc}{section}{Contacts}

\subsubsection*{Internal supervisor(s)}
\addcontentsline{toc}{subsection}{Internal supervisor(s)}

{\textit{name:}} Peter Hellinckx \\
{\textit{mail:}} peter.hellinckx@uantwerpen.be

\subsubsection*{Students}
\addcontentsline{toc}{subsection}{Students}

{\textit{name:}} Bernd Smits\\
{\textit{phone:}} +32/470 04 53 69\\
{\textit{mail:}} bernd.smits@student.uantwerpen.be

%\section{List of references}
\bibliography{ThesisProposal} {}
\addcontentsline{toc}{section}{List of references}
\bibliographystyle{ieeetr}

\end{document}